\documentclass{article}
\usepackage[english]{babel}
%\usepackage{fullpage}
\usepackage{amsfonts, amssymb, amsmath}
\usepackage{color}
\usepackage{suffix}
\usepackage[utf8]{inputenc}
\usepackage[T1]{fontenc}

%%%% Notation
%% Probability notation
\newcommand{\p}{\mathbf{P}}

\renewcommand{\Pr}[2][]{\p_{#1}\left[#2\right]} % Pr()
\WithSuffix\newcommand\Pr*[1]{\p\left[\vphantom{\hat{X}}#1\right]}

\newcommand{\pr}[2][]{\p_{#1}\,#2} % Pr
\WithSuffix\newcommand\pr*[1]{\mathbf{P}\left[\vphantom{\hat{X}}#1\right]} % Pr[]

\newcommand{\Dstr}[2][]{\mu_{#1}\left(#2\right)}
\newcommand{\dstr}[2][]{\mu_{#1}#2}

%% Miscellaneous
\newcommand{\size}[1]{\left\lvert #1\right\rvert} % | |
\WithSuffix\newcommand\size*[1]{\left\lvert\vphantom{X^Y_Z}#1\right\rvert}

\newcommand\set[1]{\left\{#1\right\}}
\WithSuffix\newcommand\set*[1]{\left\{\vphantom{X^Y_Z}#1\right\}}

\newcommand\braces[1]{\left[#1\right]}
\WithSuffix\newcommand\braces*[1]{\left[\vphantom{\overline{X^Y_Z}}#1\right]}

\newcommand{\abs}[1]{\left\lvert #1\right\rvert} % | |
\newcommand{\norm}[1]{\left\lVert #1\right\rVert} % || ||
\newcommand{\defeq}{:=}

\newcommand\suchthat{\mid}
\WithSuffix\newcommand\suchthat*{\mathrel{}\middle|\mathrel{}}

\DeclareMathOperator*{\adv}{adv}
\DeclareMathOperator*{\difference}{\Delta}

\newcommand\dist[3][]{\difference_{#1}\left(#2\,;\,#3\right)}
\WithSuffix\newcommand\dist*[3][]{\difference_{#1}\left(\vphantom{X^Y_Z}#2\,;\,#3\right)}

%%%% Editorial Commands
\newcommand{\todo}[1]{\textcolor{red}{TODO~#1}}

\usepackage{hyperref}
\usepackage{subfig}
\usepackage{algorithm}
\usepackage{algpseudocode}
\usepackage{booktabs}


\title{GCM Security Bounds}
\author{Atul Luykx}

\begin{document}
\maketitle
\section{GCM use in TLS}

\begin{table}[H]
  \centering
  \caption{Lower bounds on the maximum amount of data that can be processed under one key, given a certain attack success probability threshold, and assuming at most $2^{60}$ verification attempts were made. The numbers do not change significantly if the number of verification attempts is lowered. Equation \ref{sigmabound} was used for the calculations. Note that the calculations were made assuming that the underlying block cipher is AES. They change significantly if a block cipher is used with smaller block size.}
  \begin{tabular}{cc}
    \toprule
    Attack Success Probability  & Max data\\
    \cmidrule{1-2}
    $2^{-30}$ & 6 petabytes\\
    $2^{-20}$ & 200 petabytes \\
    $2^{-10}$ & 6400 petabytes\\
    \bottomrule
  \end{tabular}
\end{table}

\section{Computations}
\begin{table}[H]
  \centering
  \caption{Notation}
  \begin{tabular}{ll}
    \toprule
    Parameter & Description\\
    \cmidrule{1-2}
    $n$ & Block size\\
    $\tau$ & Tag size\\
    $\ell$ & maximum input length in blocks\\
    $\sigma$ & total plaintext length in blocks\\
    $q$ & number of encryption queries\\
    $v$ & number of verification attempts\\
    \bottomrule
  \end{tabular}
\end{table}
Combining Corollary 3 and equation 22 from Iwata et al.'s paper~\cite{GCMLNCS,cryptoeprint:2012:438}, we get the following AE security bound for GCM with nonce length 96 bits:
\begin{equation}
  \frac{0.5(\sigma+q+1)^2}{2^{n}} + \frac{v(\ell+1)}{2^{\tau}}\cdot\delta_n(\sigma+q+v+1)\,,
\end{equation}
where
\begin{equation}
  \delta_n(a) = \frac{1}{\left(1 - \frac{a-1}{2^n}\right)^{a/2}}\,.
\end{equation}

For GCM used in TLS, the nonce length is 96 bits, and $n$ and $\tau$ are 128 bits. Since the record size is $2^{14}$ bytes, $\ell$ has a value of $2^{10}$.

Assume that $\sigma+q+v+1\le 2^{64}$, then as pointed out by Iwata et al.\ , $\delta_n(\sigma+q+v+1) \le 2$, and we get
\begin{equation}
  \frac{(\sigma+q+1)^2}{2^{129}} + v\cdot\frac{2^{11}+2}{2^{128}}\,.
\end{equation}

If we want to bound adversarial advantage by $\epsilon$, we get
\begin{equation}
  \sigma+q\le \left(2^{129}\cdot\epsilon - v(2^{12}+4)\right)^{1/2}-1\,,
\end{equation}
and since $q\le\sigma$, we get
\begin{equation}\label{sigmabound}
  \sigma\le 0.5\cdot\left(2^{129}\cdot\epsilon - v(2^{12}+4)\right)^{1/2}-0.5\,.
\end{equation}

\bibliographystyle{plain}
\bibliography{main.bib}

\end{document}
